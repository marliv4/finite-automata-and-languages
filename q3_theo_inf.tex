\documentclass{article}
\usepackage{ngerman, amsmath, xcolor, cancel, hyperref}
\author{Marko Livajusic}
\date{\today}
\title{Theoretische Informatik: Endliche Automaten, Formale Sprachen und Grammatiken}
\hypersetup{
    colorlinks,
    citecolor=blue,
    filecolor=blue,
    linkcolor=blue,
    urlcolor=blue
}

\newcommand{\ra}{\rightarrow}
\begin{document}
\maketitle
\tableofcontents

\section{Q3.2: Deterministische Endliche Automaten (DEAs)}
\subsection{Transduktor}
\subsection{Akzeptor}
\subsection{Mealy- und Mooreautomat (irrelevant)}
\subsection{Minimerung von DEAs}
\section{Q3.2: Nichtdeterministische Endliche Automaten (NEAs)}
\subsection{$\epsilon-$NEAs}
\subsubsection{$\epsilon$-NEA $\rightarrow$ NEA (6 Konstruktionsregeln)}
\subsubsection{$\epsilon$-NEA $\rightarrow$ DEA (Potenzmengenkonstruktion)}
\subsection{NEA $\rightarrow$ DEA (Potenzmengenkonstruktion)}
\section{Q3.2: RegEx}
\subsection{RegEx $\rightarrow$ $\epsilon$-NEA}

\subsection{Formale Sprachen}
\subsubsection{Reguläre Sprachen}
Eine Sprache $L$ ist dann \textit{regulär}, wenn diese sich darstellen lässt mithilfe eines:
\begin{enumerate}
    \item deterministischen endlichen Automatens
    \item regulären Ausdrucks
\end{enumerate}

\section{Q3.2: Grammatiken}
Eine Grammatik $G$ ist ein 4-Tupel $G=\{N,T,P,S\}$, wobei
\begin{itemize}
    \item $N$ das \textbf{Nichtterminalalphabet}
    \item $T$ das \textbf{Terminalalphabet}
    \item $P$ die \textbf{Produktionen}
    \item $S$ das \textbf{Startsymbol} ist.
\end{itemize}
\subsection{Typ 3 Grammatik (regulär)}
Eine Grammatik $G$ ist dann \textit{regulär}, wenn in den Produktionen $P$
\begin{itemize}
    % https://de.wikipedia.org/wiki/Regul%C3%A4re_Grammatik#Definition
    \item links ein Nichtterminal und rechts ein oder mehrere Terminale vorkommen gefolgt von maximal einem Nichtterminal
\end{itemize}
\section{Ableitung}
Gegeben sei folgende Grammatik:
\begin{align*}
    T &= \{x,y,z\}\\
    N &= \{S, M, A, V\}\\
    P &= \{\\
        &S\rightarrow A|M|V\\
        &A\rightarrow (S+S)\\
        &M\rightarrow (S\cdot S)\\
        &V\rightarrow x|y|z\\
    &\}
\end{align*}
Wie wird das Wort $(x\cdot (y+z))$ gebildet?
\begin{align*}
    S\Rightarrow M\Rightarrow (S\cdot S)\\
    \Rightarrow (v\cdot S)\Rightarrow (x\cdot S)\Rightarrow (x\cdot A)\Rightarrow\\ (x\cdot (S+S))\Rightarrow (x\cdot (v + S))\Rightarrow (x\cdot (y + S))\Rightarrow (x\cdot (y + v))\Rightarrow (x\cdot (y + z))
\end{align*}
\subsection{Ableitungsbaum}
Dies kann man auch mit einem Ableitungsbaum darstellen:
\section{Kontextfreie Sprachen}
Gegeben sei folgende kontextfreie Grammatik:
\begin{align*}
    N&=\{A,B,S\}\\
    T&=\{a,b,\epsilon\}\\
    S&=S\\
    P&=\{\\
        &S\ra AB\\
        &S\ra ABA\\
        &A\ra aA\\
        &A\ra a\\
        &B\ra Bb\\
        &B\ra \epsilon\\
    \}
\end{align*}
\subsection{Chomsky-Normalform (klausurrelevant, abitur-irrelevant)}
\label{cnf}
\subsubsection{1. $\epsilon$-Elimination}
Zuerst wird $B\ra \epsilon$ entfernt. Die aktualisierte Grammatik lautet:
\begin{align*}
    N&=\{A,B,S\}\\
    T&=\{a,b,\epsilon\}\\
    S&=S\\
    P=\{\\
    &S\ra AB\\
    &\mathbf{S\ra A}\\
    &\mathbf{S\ra AA}\\
    &A\ra aA\\
    &A\ra a\\
    &B\ra b\\
    \}
\end{align*}
\subsubsection{2. Elimination von Kettenregeln}
Die Kettenregeln, d.h. überall da, wo ein Nichtterminal auf ein anderes Nichtterminal folgt, d.h. $S\ra A$, werden entfernt.
\begin{align*}
    N&=\{A,B,S\}\\
    T&=\{a,b,\epsilon\}\\
    S&=S\\
    P=\{\\
    &S\ra AB\\
    &S\ra AA\\
    &A\ra aA\\
    &A\ra a\\
    &B\ra b\\
    \}
\end{align*}
\subsubsection{3. Separation von Terminalzeichen}
Jedes Terminal wird durch ein Nichtterminal ersetzt:
\begin{align*}
    N&=\{A,B,S\}\\
    T&=\{a,b,\epsilon\}\\
    S&=S\\
    P=\{\\
    &S\ra AB\\
    &S\ra aA \quad |\ V_a = a\\
    &S\ra V_{a}A\\
    &S\ra a\\
    &S\ra ABA\\
    &S\ra AA\\
    &A\ra a\\
    &B\ra BV_b\\
    &B\ra b\\
    &V_a\ra a\\
    &V_b\ra b\\
    \}
\end{align*}
\subsubsection{4. Elimination von mehrelementigen Nonterminalketten}
In diesem Schritt wird die Anzahl von Nichtterminalen auf 2 reduziert, d.h. $S\ra ABA$ wird zu $S\ra S_2A$, wobei $S_2$ als $S_2\ra AB$ definiert wird.
\begin{align*}
    N&=\{A,B,S\}\\
    T&=\{a,b,\epsilon\}\\
    S&=S\\
    P=\{\\
    &S\ra AB\\
    &S\ra V_{a}A\\
    &S\ra a\\
    &\mathbf{S\ra S_2A}\\
    S_2\ra AB
    &S\ra AA\\
    &A\ra a\\
    &B\ra BV_b\\
    &B\ra b\\
    &V_a\ra a\\
    &V_b\ra b\\
    \}
\end{align*}
\subsection{CYK-Algorithmus (klausurrelevant, abitur-irrelevant)}
Mit dem CYK-Algorithmus lässt sich sagen, ob ein Wort $\omega$ in einer kontextfreien Sprache liegt. Voraussetzung für den CYK-Algorithmus ist die Chomsky-Normalform (\ref{cnf}).
\end{document}