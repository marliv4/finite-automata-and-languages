\begin{titlepage}\centering
    \vspace*{\fill}
    \LARGE Teil 2: Formale Sprachen
    \vspace*{\fill}
\end{titlepage}
\chapter{Formale Sprachen}
%
\section{Reguläre Sprachen}
\begin{definition}
    Eine Sprache $L$ ist dann regulär, wenn diese sich darstellen lässt mithilfe eines:
    \begin{enumerate}
        \item nichtdeterministischen endlichen Automatens
        \item deterministischen endlichen Automatens
        \item regulären Ausdrucks.
    \end{enumerate}
\end{definition}
\section{Q3.2: Grammatiken}
\begin{definition}
    Eine Grammatik $G$ ist ein 4-Tupel $G=\{N,T,P,S\}$, wobei
    \begin{itemize}
        \item $N$ das \textbf{Nichtterminalalphabet}
        \item $T$ das \textbf{Terminalalphabet}
        \item $P$ die \textbf{Produktionen}
        \item $S$ das \textbf{Startsymbol} ist.
    \end{itemize}
\end{definition}
\subsection{Typ 3 Grammatik (regulär)}
Eine Grammatik $G$ ist dann \textit{regulär}, wenn in den Produktionen $P$
\begin{itemize}
    % https://de.wikipedia.org/wiki/Regul%C3%A4re_Grammatik#Definition
    \item links ein Nichtterminal und rechts ein oder mehrere Terminale vorkommen gefolgt von maximal einem Nichtterminal
\end{itemize}
\section{Ableitung}
Gegeben sei folgende Grammatik:
\begin{align*}
    &T=\{x,y,z\}\\
    &N=\{S, M, A, V\}\\
    &P=\{\\
    S&\rightarrow A|M|V\\
    A&\rightarrow (S+S)\\
    M&\rightarrow (S\cdot S)\\
    V&\rightarrow x|y|z\\
    &\}
\end{align*}
Wie wird das Wort $(x\cdot (y+z))$ gebildet?
\begin{align*}
    S\Rightarrow M\Rightarrow (S\cdot S)\\
    \Rightarrow (v\cdot S)\Rightarrow (x\cdot S)\Rightarrow (x\cdot A)\Rightarrow\\ (x\cdot (S+S))\Rightarrow (x\cdot (v + S))\Rightarrow (x\cdot (y + S))\Rightarrow (x\cdot (y + v))\Rightarrow (x\cdot (y + z))
\end{align*}
\subsection{Ableitungsbaum}
Dies kann man auch mit einem Ableitungsbaum darstellen:
\subsection{Syntaxdiagramme: Regeln}
\begin{enumerate}
    \item $1$ Syntaxdiagramm $\hat{=}\ 1\ $ Produktionsregel, wobei das Syntaxdiagramm der Name der Produktionsregel ist
    \item Nichtterminale: eckig
    \item Terminale: rund
\end{enumerate}

\section{Kontextfreie Sprachen}
Gegeben sei folgende kontextfreie Grammatik:
\begin{align*}
    N&=\{A,B,S\}\\
    T&=\{a,b,\epsilon\}\\
    S&=S\\
    P&=\{\\
        &S\to AB\\
        &S\to ABA\\
        &A\to aA\\
        &A\to a\\
        &B\to Bb\\
        &B\to \epsilon\\
        \}
    \end{align*}
\subsection{Chomsky-Normalform}
\label{cnf}
\begin{definition}
    Die Chomsky-Normalform ist eine Normalform für kontextfreie Grammatiken und ist die Voraussetzung für den \nameref{cyk}.
    % \footnote{Sowohl die CNF als auch der CYK-Algorithmus sind zwar klausurrelevant, jedoch abitur-irrelevant}.
\end{definition}

% \subsection{CNF an einem Beispiel}
Gegeben sei folgende Grammatik, die in die Chomsky-Normalform gebracht werden sollte:
\begin{align*}
    G&=(N,T,P,S)\\
    N&=\{A,B\}\\
    T&=\{0\}\\
    P&=\{\\
    &A\to BAB|B|\epsilon\\
    &B\to 00|\epsilon    
    \}
\end{align*}
Um eine Grammatik $G$ in die Chomsky-Normalform zu bringen, müssen 4 Regeln befolgt werden:
\begin{enumerate}
    \item Wähle ein neues Startsymbol.
    \item Eliminiere $\epsilon$
    \item Eliminiere \textit{unit rules}, d.h. Nichtterminal auf ein Nichtterminal, bspw. $S\to A$
    \item Verändere alle Regeln, wo mehr als ein Terminal vorkommt, bspw. $S\to 00$
    \item Verändere alle Regeln, wo mehr als zwei Nichtterminale vorkommen, bspw. $S\to AB$
\end{enumerate}
\iffalse
\subsubsection[Epsilon-Elimination]{$\epsilon$-Elimination}
Wenn das Startsymbol $S$ die Produktion $S\to \epsilon$ hat, so wird ein neues Startsymbol mit $S_1 \to S$ erstellt und das neue Startsymbol hat keinen $\epsilon$ mehr in der Produktion.\par

\begin{align*}
    G&=(N,T,P,S)\\
    N&=\{S,A\}\\
    T&=\{a\}\\
    S&\to ASA|S|\epsilon\\
    A&\to aa|\epsilon
\end{align*}
Nach der 1. Regel:
\begin{align*}
    S&\to S_1\\
    S_1&\to ASA|S_1|\epsilon\\
    A&\to aa|\epsilon
\end{align*}

Nach der 2. Regel:
\begin{align*}
    S&\to S_1\\
    S_1&\to ASA|S_1\\
    S_1&\to AS|S_1\\
    S_1&\to AA|S_1\\
    S_1&\to SA|S_1\\
    A&\to aa
\end{align*}
\subsubsection{1. $\epsilon$-Elimination}
Zuerst wird $B\to \epsilon$ entfernt. Die aktualisierte Grammatik lautet:
\begin{align*}
    N&=\{A,B,S\}\\
    T&=\{a,b,\epsilon\}\\
    S&=S\\
    P=\{\\
    &S\to AB\\
    &\mathbf{S\to A}\\
    &\mathbf{S\to AA}\\
    &A\to aA\\
    &A\to a\\
    &B\to b\\
    \}
\end{align*}
\subsubsection{2. Elimination von Kettenregeln}
Die Kettenregeln, d.h. überall da, wo ein Nichtterminal auf ein anderes Nichtterminal folgt, d.h. $S\to A$, werden entfernt.
\begin{align*}
    N&=\{A,B,S\}\\
    T&=\{a,b,\epsilon\}\\
    S&=S\\
    P=\{\\
    &S\to AB\\
    &S\to AA\\
    &A\to aA\\
    &A\to a\\
    &B\to b\\
    \}
\end{align*}
\subsubsection{3. Separation von Terminalzeichen}
Jedes Terminal wird durch ein Nichtterminal ersetzt:
\begin{align*}
    N&=\{A,B,S\}\\
    T&=\{a,b,\epsilon\}\\
    S&=S\\
    P=\{\\
    &S\to AB\\
    &S\to aA \quad |\ V_a = a\\
    &S\to V_{a}A\\
    &S\to a\\
    &S\to ABA\\
    &S\to AA\\
    &A\to a\\
    &B\to BV_b\\
    &B\to b\\
    &V_a\to a\\
    &V_b\to b\\
    \}
\end{align*}
\subsubsection{4. Elimination von mehrelementigen Nonterminalketten}
In diesem Schritt wird die Anzahl von Nichtterminalen auf 2 reduziert, d.h. $S\to ABA$ wird zu $S\to S_2A$, wobei $S_2$ als $S_2\to AB$ definiert wird.
\begin{align*}
    N&=\{A,B,S\}\\
    T&=\{a,b,\epsilon\}\\
    S&=S\\
    P=\{\\
    &S\to AB\\
    &S\to V_{a}A\\
    &S\to a\\
    &\mathbf{S\to S_2A}\\
    S_2\to AB
    &S\to AA\\
    &A\to a\\
    &B\to BV_b\\
    &B\to b\\
    &V_a\to a\\
    &V_b\to b\\
    \}
\end{align*}
\fi
\subsection{CYK-Algorithmus}
\label{cyk}
Mit dem CYK-Algorithmus lässt sich sagen, ob ein Wort $\omega$ in einer kontextfreien Sprache liegt. Die Voraussetzung für den CYK-Algorithmus ist die \nameref{cnf}.
\begin{bsp}
    Sei $G$ eine Grammatik mit Produktionsregeln $P$, die definiert sind als:
    \begin{align*}
        &S\to BC|AC|BA\\
        &A\to AA|BB|a\\
        &B\to BA|b\\
        &C\to AC|c
    \end{align*}
    Nun bestimme man, ob das Wort $ababac$ in $L(G)$ liegt.
\end{bsp}

\setlength{\arraycolsep}{0pt}
$\begin{NiceArray}
  [columns-width=15mm, corners = NE, last-row, code-for-last-row = \color{blue}, hvlines]
  { cccccc }
  \RowStyle[nb-rows=6]{\rule[-2mm]{0pt}{15mm}}
    \NotEmpty \\
    \NotEmpty   & \NotEmpty \\
    \NotEmpty   & \NotEmpty & \NotEmpty \\
    \NotEmpty & \NotEmpty & \NotEmpty & \NotEmpty \\
    \NotEmpty   & \NotEmpty & \NotEmpty & \NotEmpty & \NotEmpty \\
    \NotEmpty           &           &           &           & \NotEmpty & \NotEmpty \\
    % \{A\}       & \{A\}     & \{B\}     & \{B\}     & \{B\} & \{C\} \\
    a           & b         & a         & b         & a     & c
\CodeAfter
  \tikz
    \foreach \x in {1,...,6}
       \foreach \y in {\x,...,6} 
         { \node at (\y-|\x) [anchor=north west] {}; } ;
        %  { \node at (\y-|\x) [anchor=north west] {\small $\x, \the\numexpr 6-\y+\x\relax$ } ; } ;
\end{NiceArray}$\par
Die unterste Zeile ist die 1. Zeile. Fangen wir (von links) mit dem ersten Feld der ersten Zeile, so sehen wir, dass ein Nichtterminalsymbol gesucht ist, welches das Wort \textit{a} ableitet. Schaut man auf die Grammatik, so sieht man, dass lediglich die Produktionsregel $A$ das Wort \textit{a} ableitet, weshalb sie in das untere Feld eingetragen wird:
$\begin{NiceArray}
  [columns-width=15mm, corners = NE, last-row, code-for-last-row = \color{blue}, hvlines]
  { cccccc }
  \RowStyle[nb-rows=6]{\rule[-2mm]{0pt}{15mm}}
    \NotEmpty \\
    \NotEmpty   & \NotEmpty \\
    \NotEmpty   & \NotEmpty & \NotEmpty \\
    \NotEmpty & \NotEmpty & \NotEmpty & \NotEmpty \\
    \NotEmpty   & \NotEmpty & \NotEmpty & \NotEmpty & \NotEmpty \\
    \NotEmpty           &           &           &           & \NotEmpty & \NotEmpty \\
    \{A\}       & \{A\}     & \{B\}     & \{B\}     & \{B\} & \{C\} \\
    a           & b         & a         & b         & a     & c
\CodeAfter
  \tikz
    \foreach \x in {1,...,6}
       \foreach \y in {\x,...,6} 
         { \node at (\y-|\x) [anchor=north west] {}; } ;
        %  { \node at (\y-|\x) [anchor=north west] {\small $\x, \the\numexpr 6-\y+\x\relax$ } ; } ;
\end{NiceArray}$\par