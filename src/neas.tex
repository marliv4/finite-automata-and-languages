\newcommand{\enea}{$\epsilon$-NEA}
\newcommand{\ehu}{$\epsilon$\text{-Hülle}\ }

\chapter{Nichtdeterministische Endliche Automaten}
\section[Epsilon-NEAs]{\enea s}
\begin{definition}
    Ein \enea\ ist ein Akzeptor, der $\epsilon$-Übergänge besitzt und deshalb mit dem leeren Wort Zustände wechseln kann.
\end{definition}
% Umwandlung
\subsection[Epsilon-NEA zu NEA]{\enea $\to$ NEA}
Gegeben sei folgendes Zustandsdiagramm eines \enea, welches in einen NEA umgewandelt werden soll:
\begin{figure}[H]
    \centering
    \begin{transitiongraph}[fa]
        \state[s]{q0}{0}{0}
        \state{q1}{40}{0}
        \state{q2}{80}{0}
        \state[f]{q3}{120}{0}
        \transition{q0}{q1}{0}
        \transition{q1}{q1}{1}
        \transition{q1}{q2}{}
        \transition{q2}{q2}{0}
        \transition{q2}{q3}{1}
    \end{transitiongraph}
    % \caption{eps_nea_to_nea}
    % \label{graph:eps_nea_to_nea}
\end{figure}
Zuerst wird eine leere Übergangstabelle erstellt:
\begin{table}[H]
    \centering
    \begin{tabular}{|l|l|l|}
    \hline
    Zustand & 0 & 1 \\ \hline
    $q_0$   &   &   \\ \hline
    $q_1$   &   &   \\ \hline
    $q_2$   &   &   \\ \hline
    $q_3$   &   &   \\ \hline
    \end{tabular}
\end{table}
Danach wird für jedes Eingabesymbol eine Tabelle mit der $\epsilon$-Hülle erstellt:
\begin{table}[H]
    \centering
    \begin{tabular}{|l|l|l|l|}
    \hline
    Zustand & $\epsilon\mbox{*}$ & $0$   & $\epsilon\mbox{*}$ \\ \hline
    $q_0$   &                    &       &                    \\ \hline
            &                    &       &                    \\ \hline
            &                    &       &                    \\ \hline
            &                    &       &                    \\ \hline
    \end{tabular}
\end{table}
Wie oben zu sehen ist, wird zuerst der Startzustand $q_0$ eingetragen. Danach wird die \ehu des Zustands $q_0$ berechnet und eingetragen.
\begin{definition}
    Eine \ehu ist die Menge aller Zustände, die ein Zustand $q_n$ mit dem leeren Wort $\epsilon$ erreichen kann.
\end{definition}
Da im vorigen Beispiel $q_0$ mit dem leeren Wort keinen anderen Zustand als sich selbst erreichen kann, wird für dessen \ehu $q_0$ eingetragen.\par
Die nächte Spalte steht für den Zustand, der erreicht wird, wenn bei $q_0$ das Eingabesymbol $0$ eingegeben wird. Dies ist in diesem Beispiel der Zustand $q_1$:
\begin{table}[H]
    \centering
    \begin{tabular}{|l|l|l|l|}
    \hline
    Zustand & $\epsilon\mbox{*}$ & $0$   & $\epsilon\mbox{*}$ \\ \hline
    $q_0$   & $q_0$              & $\mathbf{q_1}$ &                    \\ \hline
            &                    &       &                    \\ \hline
            &                    &       &                    \\ \hline
            &                    &       &                    \\ \hline
    \end{tabular}
\end{table}
Die letzte Spalte bezieht sich auf die \ehu des Zustands aus der mittleren Spalte, welcher hier fettgedruckt steht. Die \ehu von $q_1$ ist dabei $\{q_1, q_2\}$. Diese wird ebenfalls eingetragen:
\begin{table}[H]
    \centering
    \begin{tabular}{|l|l|l|l|}
    \hline
    Zustand & $\epsilon\mbox{*}$ & $0$   & $\epsilon\mbox{*}$ \\ \hline
    $q_0$   & $q_0$              & $q_1$ & $\{q_1,q_2\}$      \\ \hline
            &                    &       &                    \\ \hline
            &                    &       &                    \\ \hline
            &                    &       &                    \\ \hline
    \end{tabular}
\end{table}
Diese \ehu $\{q_1, q_2\}$ repräsentiert dabei die Zustände, die $q_0$ bei der Eingabe von $0$ erreicht werden. Deshalb können diese in die Übergangstabelle eingetragen werden:
\begin{table}[H]
    \centering
    \begin{tabular}{|l|l|l|}
    \hline
    Zustand & 0           & 1 \\ \hline
    $q_0$   & $\{q_1,q_2\}$ &   \\ \hline
    $q_1$   &             &   \\ \hline
    $q_2$   &             &   \\ \hline
    $q_3$   &             &   \\ \hline
    \end{tabular}
\end{table}
Dieser Vorgang wird für alle Zustände durchgeführt, sowohl für die Eingabe von $0$ als auch von $1$. Die Tabellen sehen nach dem Algorithmus wie folgt aus:
\begin{table}[H]
    \centering
    \begin{tabular}{|l|l|l|l|}
    \hline
    Zustand & $\epsilon\mbox{*}$ & $0$         & $\epsilon\mbox{*}$ \\ \hline
    $\{q_0\}$   & $\{q_0\}$              & $\{q_1\}$       & $\{q_1,q_2\}$      \\ \hline
    $\{q_1\}$ &
      \begin{tabular}[c]{@{}l@{}}$\{q_1\}$\\ $\{q_1,q_2\}$\end{tabular} &
      \begin{tabular}[c]{@{}l@{}}$\emptyset$\\ $\{q_2\}$\end{tabular} &
      \begin{tabular}[c]{@{}l@{}}$\emptyset$\\ $\{q_2\}$\end{tabular} \\ \hline
    $\{q_2\}$   & $\{q_2\}$              & $\{q_2\}$       & $\{q_2\}$              \\ \hline
    $\{q_3\}$   & $\{q_3\}$              & $\emptyset$ & $\emptyset$        \\ \hline
    \end{tabular}
\end{table}
\begin{table}[H]
    \centering
    \begin{tabular}{|l|l|l|l|}
    \hline
    Zustand   & $\epsilon\mbox{*}$ & $1$         & $\epsilon\mbox{*}$ \\ \hline
    $\{q_0\}$ & $\{q_0\}$          & $\emptyset$ & $\emptyset$        \\ \hline
    $\{q_1\}$ &
      \begin{tabular}[c]{@{}l@{}}$\{q_1\}$\\ $\{q_2\}$\end{tabular} &
      \begin{tabular}[c]{@{}l@{}}$\{q_1\}$\\ $\{q_3\}$\end{tabular} &
      \begin{tabular}[c]{@{}l@{}}$\{q_1,q_2\}$\\ $\{q_3\}$\end{tabular} \\ \hline
    $\{q_2\}$ & $\{q_2\}$          & $\{q_3\}$   & $\{q_3\}$          \\ \hline
    $\{q_3\}$ & $\{q_3\}$          & $\emptyset$ & $\emptyset$        \\ \hline
    \end{tabular}
    \end{table}
\begin{table}[H]
    \centering
    \begin{tabular}{|l|l|l|}
        \hline
        Zustand & 0             & 1               \\ \hline
        $\{q_0\}$   & $\{q_1,q_2\}$ & $\emptyset$     \\ \hline
        $\{q_1\}$   & $\{q_2\}$         & $\{q_1,q_2,q_3\}$ \\ \hline
        $\{q_2\}$   & $\{q_2\}$         & $\{q_3\}$           \\ \hline
        $\{q_3\}$   & $\emptyset$   & $\emptyset$     \\ \hline
    \end{tabular}
\end{table}
Noch sollen die Endzustände ermittelt werden. Zu den Endzuständen gehört der Endzustand aus dem \enea und die Zustände, die durch das leere Wort $\epsilon$ in den ursprünglichen Endzustand gelangen können. Deshalb wird in diesem Fall nur $q_3$ der Endzustand. Gezeichnet sieht das neue Zustandsdiagramm wie folgt aus:
\begin{figure}[H]
    \centering
    \begin{transitiongraph}[fa]
        \state[s]{q0}{0}{0}
        \state[f]{q3}{80}{-40}
        \state{q2}{40}{-80}
        \state{o}{40}{-40}
        \state{q1}{120}{-40}
        \transition{q0}{q1}{0}
        \transition{q0}{q2}{0}
        \transition{q0}{o}{1}
        \transition{q3}{o}{0;1}
        \transition{q2}{q2}{0}
        \transition{q2}{q3}{1}
        \transition{q1}{q2}{0;1}
        \transition{q1}{q1}{1}
        \transition{q1}{q3}{1}
    \end{transitiongraph}
    \caption{Der neue NEA, ohne $\epsilon$-Übergänge.}
    \label{graph:jgd}
\end{figure}
\glqq o\grqq\ steht hier für die leere Menge $\emptyset$.
%
\subsection[Epsilon-NEA zu DEA]{\enea $\to$ DEA}
Es sei folgendes Zustandsdiagramm eines $\epsilon$-NEAs gegeben:\par
\begin{figure}[H]
    \centering
    \begin{transitiongraph}[fa]
        \state[s]{q0}{0}{0}
        \state[f]{q2}{60}{0}
        \state{q4}{120}{-60}
        \state{q1}{0}{-60}
        \state{q3}{120}{0}
        \transition[line=left]{q0}{q1}{A}
        \transition{q0}{q2}{}
        \transition{q2}{q3}{;B}
        \transition[line=left]{q2}{q4}{A}
        \transition[line=left]{q4}{q2}{B}
        \transition[line=left]{q1}{q0}{A}
        \transition{q3}{q4}{A}
    \end{transitiongraph}
    % \caption{epsNEA}
    \label{graph:epsNEA}
\end{figure}
Die Umwandlung in ein DEA geschieht wie üblich mit der Potenzmengenkonstruktion:
\begin{table}[H]
\centering
\begin{tabular}{|l|l|l|}
\hline
Zustand         & A           & B               \\ \hline
$\to \{q_0\}$       & $\{q_1,q_4\}$ & $\{q_3\}$           \\ \hline
$\{q_1,q_4\}$     & $\{q_0\}$     & $\{q_2\mbox{*}\}$ \\ \hline
$\{q_3\}$         & $\{q_4\}$     & $\emptyset$     \\ \hline
$\{q_2\mbox{*}\}$ & $\{q_4\}$     & $\{q_3\}$         \\ \hline
$\{q_4\}$         & $\emptyset$ & $\{q_2\}$           \\ \hline
$\emptyset$     & $\emptyset$ & $\emptyset$     \\ \hline
\end{tabular}
\end{table}
Anschlißend wird das neue Zustandsdiagramm des DEAs gezeichnet. \textit{qE} repräsentiert dabei die leere Menge $\emptyset$.\par
\begin{figure}[H]
    \centering
    \begin{transitiongraph}[fa]
        \state[s]{q0}{0}{0}
        \state{q1q4}{0}{-40}
        \state{q4}{80}{-80}
        \state[f]{q2}{40}{-40}
        \state{qE}{120}{0}
        \state{q3}{80}{0}
        \transition[line=left]{q0}{q1q4}{A}
        \transition{q0}{q3}{B}
        \transition[line=left]{q1q4}{q0}{A}
        \transition{q1q4}{q2}{B}
        \transition[line=left]{q4}{q2}{B}
        \transition{q4}{qE}{A}
        \transition[line=left]{q2}{q4}{A}
        \transition{q2}{q3}{B}
        \transition{qE}{qE}{A;B}
        \transition{q3}{q4}{A}
        \transition{q3}{qE}{B}
    \end{transitiongraph}
    \caption{Umwandlung von $\epsilon$-NEA zu DEA. Dieser ist jedoch nicht zwangsläufig optimal bzw. minimal.}
\end{figure}
\newpage
\section[NEA zu DEA mit Potenzmengenkonstruktion]{NEA $\to$ DEA (Potenzmengenkonstruktion)}
Dieser NEA soll in einen DEA umgewandelt werden:
\begin{figure}[H]
    \centering
    \begin{transitiongraph}[fa]
        \state[s]{q0}{0}{0}
        \state{q1}{65}{0}
        \state[f]{q2}{120}{0}
        \transition{q0}{q0}{a;b}
        \transition{q0}{q1}{b}
        \transition{q1}{q2}{b}
        \transition{q2}{q2}{a;b}
    \end{transitiongraph}
    % \caption{NEA_LaTeX}
    \label{graph:NEA_LaTeX}
\end{figure}
\sloppy
\textbf{Vorgehen}: Es wird zuerst eine Übergangstabelle aufgestellt und geschaut, welche Zustände neu auftreten.
\fussy
\begin{table}[H]
\centering
\begin{tabular}{|l|l|l|}
\hline
Zustand             & $a$              & $b$                   \\
\hline
$\to \{q_0\}$           & $\{q_0\}$           & $\{q_0,q_1\}$       \\
\hline
$\{q_0,q_1\}$       & $\{q_0\}$           & $\{q_0,q_1,q_2\mbox{*}\}$ \\
\hline
$\{q_0,q_1,q_2\}\mbox{*}$ & $\{q_0,q_2\mbox{*}\}$ & $\{q_0,q_1,q_2\mbox{*}\}$   \\
\hline
$\{q_0,q_2\}\mbox{*}$      & $\{q_0,q_2\mbox{*}\}$ & $\{q_0,q_1,q_2\mbox{*}\}$ \\
\hline
\end{tabular}
\end{table}
Danach wird aus dieser Übergangstabelle der DEA gezeichnet:
\begin{figure}[H]
    \centering
    \begin{transitiongraph}[fa]
        \state[s]{q0}{0}{0}
        \state[f]{q02}{120}{-60}
        \state{q01}{0}{-60}
        \state[f]{q012}{60}{-60}
        \transition{q0}{q0}{a}
        \transition[line=left]{q0}{q01}{b}
        \transition{q02}{q02}{a}
        \transition[line=left]{q02}{q012}{b}
        \transition[line=left]{q01}{q0}{a}
        \transition{q01}{q012}{b}
        \transition[line=left]{q012}{q02}{a}
        \transition{q012}{q012}{b}
    \end{transitiongraph}
\end{figure}
