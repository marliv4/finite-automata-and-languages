\documentclass{book}
\usepackage{ngerman, amsmath, xcolor, cancel, hyperref, float, FLaAL}
\newtheorem{bsp}{Beispiel}
\newtheorem{definition}{Definition}

\author{Marko Livajusic}
\date{\today}
\title{Theoretische Informatik: Endliche Automaten, Formale Sprachen und Grammatiken}
\hypersetup{
    colorlinks,
    citecolor=blue,
    filecolor=blue,
    linkcolor=blue,
    urlcolor=blue
}

\begin{document}
\maketitle
\tableofcontents

\section{Deterministische Endliche Automaten (DEAs)}
\subsection{Transduktor}
\begin{definition}
Ein Transduktorautomat $\mathcal{T}: \{\Sigma, A, Z, z_{0}, \delta, \lambda\}$ ist ein deterministicher endlicher Automat ohne einen Endzustand.
\end{definition}
\begin{align*}
    \mathbf{\Sigma} &: \text{Eingabealphabet}\\
    \mathbf{A} &: \text{Ausgabealphabet}\\
    \mathbf{Z} &: \text{Zustandsmenge}\\
    \mathbf{z_{0}} \in Z &: \text{Startzustand}\\
    \mathbf{\delta }: \Sigma \times Z \rightarrow Z &: \text{Überführungsfunktion}\\
    \mathbf{\lambda }: \Sigma \times Z \rightarrow A\text{*} &: \text{Ausgabefunktion}
\end{align*}

\subsection{Akzeptor}
\begin{definition}
Ein Akzeptor $\mathcal{A}: \{\Sigma, Z, z_{0}, \delta, F\}$ ist ein deterministicher endlicher Automat, der die Eingabe überprüft und keine Ausgabe besitzt. Er lässt sich wie folgt beschreiben:
\end{definition}
    \begin{align*}
    \Sigma &: \text{Eingabealphabet}\\
    Z &: \text{Zustandsmenge}\\
    z_{0} &: \text{Startzustand}\\
    \delta &: \text{Überführungsfunktion}\\
    F &: \text{Endzustandsmenge}
\end{align*}

\subsection{Mealy- und Mooreautomat (irrelevant)}
\subsection{Minimerung von DEAs}
\section{Q3.2: Nichtdeterministische Endliche Automaten (NEAs)}
\subsection{$\epsilon-$NEAs}
\subsubsection{$\epsilon$-NEA $\rightarrow$ NEA (6 Konstruktionsregeln)}
\subsubsection{$\epsilon$-NEA $\rightarrow$ DEA (Potenzmengenkonstruktion)}

\subsection{NEA $\rightarrow$ DEA (Potenzmengenkonstruktion)}
\begin{figure}
    \centering
    \begin{transitiongraph}[fa]
        \state[s]{q0}{0}{0}
        \state{q1}{65}{0}
        \state[f]{q2}{120}{0}
        \transition{q0}{q0}{a;b}
        \transition{q0}{q1}{b}
        \transition{q1}{q2}{b}
        \transition{q2}{q2}{a;b}
    \end{transitiongraph}
    % \caption{NEA_LaTeX}
    \label{graph:NEA_LaTeX}
\end{figure}
\textbf{Vorgehen}: Es wird zuerst eine Übergangstabelle aufgestellt und geschaut, welche Zustände neu auftreten.
\begin{table}[]
\begin{tabular}{|l|l|l|}
\hline
Zustand             & $a$              & $b$                   \\
\hline
$\to q_0$           & $q_0$           & $\{q_0,q_1\}$       \\
\hline
$\{q_0,q_1\}$       & $q_0$           & $\{q_0,q_1,q_2\mbox{*}\}$ \\
\hline
$\{q_0,q_1,q_2\}\mbox{*}$ & $\{q_0,q_2\mbox{*}\}$ & $\{q_0,q_1,q_2\mbox{*}\}$   \\
\hline
$\{q_0,q_2\}\mbox{*}$      & $\{q_0,q_2\mbox{*}\}$ & $\{q_0,q_1,q_2\mbox{*}\}$ \\
\hline
\end{tabular}
\end{table}\\
Danach wird aus dieser Übergangstabelle der DEA gezeichnet:\\
\begin{figure}[H]
    \centering
    \begin{transitiongraph}[fa]
        \state[s]{q0}{0}{0}
        \state[f]{q02}{120}{-60}
        \state{q01}{0}{-60}
        \state[f]{q012}{60}{-60}
        \transition{q0}{q0}{a}
        \transition[line=left]{q0}{q01}{b}
        \transition{q02}{q02}{a}
        \transition[line=left]{q02}{q012}{b}
        \transition[line=left]{q01}{q0}{a}
        \transition{q01}{q012}{b}
        \transition[line=left]{q012}{q02}{a}
        \transition{q012}{q012}{b}
    \end{transitiongraph}
    % \caption{NEA_LaTeX}
    \label{graph:NEA_LaTeX}
\end{figure}

\section{Reguläre Ausdrücke}
\subsection{RegEx $\rightarrow$ $\epsilon$-NEA}
\end{document}