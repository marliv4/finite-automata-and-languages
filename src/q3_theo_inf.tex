\documentclass{book}
\usepackage{ngerman, amsmath, xcolor, cancel, hyperref, float, microtype, FLaAL}
\newtheorem{bsp}{Beispiel}
\newtheorem{definition}{Definition}
% \usepackage[showframe]{geometry}

\author{Marko Livajusic}
\date{\today}
\title{Theoretische Informatik: Endliche Automaten, Formale Sprachen und Grammatiken}
\hypersetup{
    colorlinks,
    citecolor=blue,
    filecolor=blue,
    linkcolor=blue,
    urlcolor=blue
}

\begin{document}
\maketitle
\tableofcontents
\newpage
\newpage
\sloppy
\section{Deterministische Endliche Automaten (DEAs)}
\fussy
\subsection{Transduktor}
\begin{definition}
Ein Transduktorautomat $\mathcal{T}: \{\Sigma, A, Z, z_{0}, \delta, \lambda\}$ ist ein deterministicher endlicher Automat ohne einen Endzustand.
\end{definition}
\begin{align*}
    \mathbf{\Sigma} &: \text{Eingabealphabet}\\
    \mathbf{A} &: \text{Ausgabealphabet}\\
    \mathbf{Z} &: \text{Zustandsmenge}\\
    \mathbf{z_{0}} \in Z &: \text{Startzustand}\\
    \mathbf{\delta }: \Sigma \times Z \rightarrow Z &: \text{Überführungsfunktion}\\
    \mathbf{\lambda }: \Sigma \times Z \rightarrow A\text{*} &: \text{Ausgabefunktion}
\end{align*}

\subsection{Akzeptor}
\begin{definition}
Ein Akzeptor $\mathcal{A}: \{\Sigma, Z, z_{0}, \delta, F\}$ ist ein deterministicher endlicher Automat, der die Eingabe überprüft und keine Ausgabe besitzt. Er lässt sich wie folgt beschreiben:
\end{definition}
    \begin{align*}
    \Sigma &: \text{Eingabealphabet}\\
    Z &: \text{Zustandsmenge}\\
    z_{0} &: \text{Startzustand}\\
    \delta &: \text{Überführungsfunktion}\\
    F &: \text{Endzustandsmenge}
\end{align*}

\subsection{Mealy-Automat (irrelevant)}
\begin{definition}
    Ein Mealy-Automat ist ein Transduktor, dessen Ausgabe von der \textbf{Überführungsfunktion} $\delta$ und vom aktuellen \textbf{Zustand} $z_n$ abhängig ist.
\end{definition}
\subsection{Moore-Automat}
\begin{definition}
    Ein Moore-Automat ist ein Transduktor, dessen Ausgabe vom aktuellen \textbf{Zustand} $z_n$ abhängig ist.
\end{definition}
\subsection{Minimierung von DEAs}
Zu minimieren sei folgender DEA:\par
\begin{figure}[H]
    \centering
    \begin{transitiongraph}[fa]
        \state[s]{q0}{0}{-23.333}
        \state{q1}{53.333}{0}
        \state{q2}{50}{-53.333}
        \state[f]{q3}{120}{-23.333}
        \transition{q0}{q1}{0}
        \transition{q0}{q2}{1}
        \transition[line=left]{q1}{q2}{0}
        \transition{q1}{q3}{1}
        \transition[line=left]{q2}{q1}{0}
        \transition{q2}{q3}{1}
        \transition{q3}{q3}{0;1}
    \end{transitiongraph}
\end{figure}
Diagonale als äquivalent markieren:
\begin{table}[H]
    \centering
    \begin{tabular}{|l|l|l|l|l|}
    \hline
    Zustand    & $q_0$       & $q_1$    & $q_2$    & $q_3$       \\ \hline
    $q_0$      & $\equiv$ &          &          &          \\ \hline
    $q_1$      &          & $\equiv$ &          &          \\ \hline
    $q_2$      &          &          & $\equiv$ &          \\ \hline
    $q_3$      &          &          &          & $\equiv$ \\ \hline
    \end{tabular}
\end{table}
Felder, wo ein Zustand auf einen Endzustand trifft, streichen
\begin{table}[H]
    \centering
    \begin{tabular}{|l|l|l|l|l|}
    \hline
    Zustand    & $q_0$    & $q_1$    & $q_2$    & $q_3$    \\ \hline
    $q_0$      & $\equiv$ &          &          &          \\ \hline
    $q_1$      &          & $\equiv$ &          &          \\ \hline
    $q_2$      &          &          & $\equiv$ &          \\ \hline
    $q_3$      &   X      &   X      &    X     & $\equiv$ \\ \hline
    \end{tabular}
\end{table}
Eine Übergangstabelle mit übrigen Zuständen erstellen. Die Zustandspaare, die auf einen bereits gestrichenen Zustandspaar abgebildet werden, streichen
\begin{table}[H]
    \centering
    \begin{tabular}{|l|l|l|}
    \hline
    Zustand      & 0           & 1           \\ \hline
    $\color{red}(q_0,q_1)$  & $(q_1,q_2)$ & $\mathbf{(q_2,q_3)}$ \\ \hline
    $\color{red}(q_0,q_2)$  & $(q_1,q_1)$ & $\mathbf{(q_2,q_3)}$ \\ \hline
    % $\color{red}(q_0,q_3)$  & $(q_1,q_3)$ & $\mathbf{(q_2,q_3)}$ \\ \hline
    $(q_1,q_2)$  & $(q_2,q_1)$ & $(q_3,q_3)$ \\ \hline
    \end{tabular}
\end{table}
Die neue Tabelle sieht dann so aus:
\begin{table}[H]
    \centering
    \begin{tabular}{|l|l|l|l|l|}
    \hline
    Zustand    & $q_0$    &     $q_1$       & $q_2$    & $q_3$    \\ \hline
    $q_0$      & $\equiv$ &                 &          &          \\ \hline
    $q_1$      &    X     &     $\equiv$    &          &          \\ \hline
    $q_2$      &    X     &                 & $\equiv$ &          \\ \hline
    $q_3$      &    X     &     X           &    X     & $\equiv$ \\ \hline
    \end{tabular}
\end{table}
Die leeren Felder als äquivalent markieren:
\begin{table}[H]
    \centering
    \begin{tabular}{|l|l|l|l|l|}
    \hline
    Zustand    & $q_0$    &     $q_1$       & $q_2$    & $q_3$    \\ \hline
    $q_0$      & $\equiv$ &                 &          &          \\ \hline
    $q_1$      &    X     &     $\equiv$    &          &          \\ \hline
    $q_2$      &    X     &     $\equiv$    & $\equiv$ &          \\ \hline
    $q_3$      &    X     &     X           &    X     & $\equiv$ \\ \hline
    \end{tabular}
\end{table}
Spaltenweise die Zustände zusammenfassen:
\begin{figure}[H]
    \centering
    \begin{transitiongraph}[fa]
        \state[s]{q0}{0}{-12}
        \state{q1q2}{64}{0}
        \state[f]{q3}{120}{-24}
        \transition{q0}{q1q2}{0;1}
        \transition{q1q2}{q3}{1}
        \transition{q1q2}{q1q2}{0}
        \transition{q3}{q3}{0;1}
    \end{transitiongraph}
\end{figure}
\section{Q3.2: Nichtdeterministische Endliche Automaten (NEAs)}
\subsection{$\epsilon-$NEAs}
\subsubsection{$\epsilon$-NEA $\rightarrow$ NEA (6 Konstruktionsregeln)}

\subsection{NEA $\rightarrow$ DEA (Potenzmengenkonstruktion)}
Dieser NEA soll in einen DEA umgewandelt werden:
\begin{figure}[H]
    \centering
    \begin{transitiongraph}[fa]
        \state[s]{q0}{0}{0}
        \state{q1}{65}{0}
        \state[f]{q2}{120}{0}
        \transition{q0}{q0}{a;b}
        \transition{q0}{q1}{b}
        \transition{q1}{q2}{b}
        \transition{q2}{q2}{a;b}
    \end{transitiongraph}
    % \caption{NEA_LaTeX}
    \label{graph:NEA_LaTeX}
\end{figure}
\sloppy
\textbf{Vorgehen}: Es wird zuerst eine Übergangstabelle aufgestellt und geschaut, welche Zustände neu auftreten.
\fussy
\begin{table}[H]
\begin{tabular}{|l|l|l|}
\hline
Zustand             & $a$              & $b$                   \\
\hline
$\to q_0$           & $q_0$           & $\{q_0,q_1\}$       \\
\hline
$\{q_0,q_1\}$       & $q_0$           & $\{q_0,q_1,q_2\mbox{*}\}$ \\
\hline
$\{q_0,q_1,q_2\}\mbox{*}$ & $\{q_0,q_2\mbox{*}\}$ & $\{q_0,q_1,q_2\mbox{*}\}$   \\
\hline
$\{q_0,q_2\}\mbox{*}$      & $\{q_0,q_2\mbox{*}\}$ & $\{q_0,q_1,q_2\mbox{*}\}$ \\
\hline
\end{tabular}
\end{table}
Danach wird aus dieser Übergangstabelle der DEA gezeichnet:\\
\begin{figure}[H]
    \centering
    \begin{transitiongraph}[fa]
        \state[s]{q0}{0}{0}
        \state[f]{q02}{120}{-60}
        \state{q01}{0}{-60}
        \state[f]{q012}{60}{-60}
        \transition{q0}{q0}{a}
        \transition[line=left]{q0}{q01}{b}
        \transition{q02}{q02}{a}
        \transition[line=left]{q02}{q012}{b}
        \transition[line=left]{q01}{q0}{a}
        \transition{q01}{q012}{b}
        \transition[line=left]{q012}{q02}{a}
        \transition{q012}{q012}{b}
    \end{transitiongraph}
\end{figure}
\section{Reguläre Ausdrücke}
\subsection{RegEx $\rightarrow$ $\epsilon$-NEA}
\end{document}