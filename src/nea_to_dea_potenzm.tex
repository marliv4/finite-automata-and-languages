\subsection{NEA $\rightarrow$ DEA (Potenzmengenkonstruktion)}
Dieser NEA soll in einen DEA umgewandelt werden:
\begin{figure}[H]
    \centering
    \begin{transitiongraph}[fa]
        \state[s]{q0}{0}{0}
        \state{q1}{65}{0}
        \state[f]{q2}{120}{0}
        \transition{q0}{q0}{a;b}
        \transition{q0}{q1}{b}
        \transition{q1}{q2}{b}
        \transition{q2}{q2}{a;b}
    \end{transitiongraph}
    % \caption{NEA_LaTeX}
    \label{graph:NEA_LaTeX}
\end{figure}
\sloppy
\textbf{Vorgehen}: Es wird zuerst eine Übergangstabelle aufgestellt und geschaut, welche Zustände neu auftreten.
\fussy
\begin{table}[H]
\begin{tabular}{|l|l|l|}
\hline
Zustand             & $a$              & $b$                   \\
\hline
$\to q_0$           & $q_0$           & $\{q_0,q_1\}$       \\
\hline
$\{q_0,q_1\}$       & $q_0$           & $\{q_0,q_1,q_2\mbox{*}\}$ \\
\hline
$\{q_0,q_1,q_2\}\mbox{*}$ & $\{q_0,q_2\mbox{*}\}$ & $\{q_0,q_1,q_2\mbox{*}\}$   \\
\hline
$\{q_0,q_2\}\mbox{*}$      & $\{q_0,q_2\mbox{*}\}$ & $\{q_0,q_1,q_2\mbox{*}\}$ \\
\hline
\end{tabular}
\end{table}
Danach wird aus dieser Übergangstabelle der DEA gezeichnet:\\
\begin{figure}[H]
    \centering
    \begin{transitiongraph}[fa]
        \state[s]{q0}{0}{0}
        \state[f]{q02}{120}{-60}
        \state{q01}{0}{-60}
        \state[f]{q012}{60}{-60}
        \transition{q0}{q0}{a}
        \transition[line=left]{q0}{q01}{b}
        \transition{q02}{q02}{a}
        \transition[line=left]{q02}{q012}{b}
        \transition[line=left]{q01}{q0}{a}
        \transition{q01}{q012}{b}
        \transition[line=left]{q012}{q02}{a}
        \transition{q012}{q012}{b}
    \end{transitiongraph}
\end{figure}