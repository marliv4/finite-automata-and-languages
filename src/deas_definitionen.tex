
\sloppy
\chapter{Deterministische Endliche Automaten}
\fussy
% \section{Deterministische Endliche Automaten (DEAs)}
\subsection{Transduktor}
\begin{definition}
Ein Transduktorautomat $\mathcal{T}: \{\Sigma, A, Z, z_{0}, \delta, \lambda\}$ ist ein deterministicher endlicher Automat ohne einen Endzustand.
\end{definition}
\begin{align*}
    \mathbf{\Sigma} &: \text{Eingabealphabet}\\
    \mathbf{A} &: \text{Ausgabealphabet}\\
    \mathbf{Z} &: \text{Zustandsmenge}\\
    \mathbf{z_{0}} \in Z &: \text{Startzustand}\\
    \mathbf{\delta }: \Sigma \times Z \rightarrow Z &: \text{Überführungsfunktion}\\
    \mathbf{\lambda }: \Sigma \times Z \rightarrow A\text{*} &: \text{Ausgabefunktion}
\end{align*}

\subsection{Akzeptor}
\begin{definition}
Ein Akzeptor $\mathcal{A}: \{\Sigma, Z, z_{0}, \delta, F\}$ ist ein deterministicher endlicher Automat, der die Eingabe überprüft und keine Ausgabe besitzt. Er lässt sich wie folgt beschreiben:
\end{definition}
    \begin{align*}
    \Sigma &: \text{Eingabealphabet}\\
    Z &: \text{Zustandsmenge}\\
    z_{0} &: \text{Startzustand}\\
    \delta &: \text{Überführungsfunktion}\\
    F &: \text{Endzustandsmenge}
\end{align*}

\subsection{Mealy-Automat (irrelevant)}
\begin{definition}
    Ein Mealy-Automat ist ein Transduktor, dessen Ausgabe von der \textbf{Überführungsfunktion} $\delta$ und vom aktuellen \textbf{Zustand} $z_n$ abhängig ist.
\end{definition}
\subsection{Moore-Automat}
\begin{definition}
    Ein Moore-Automat ist ein Transduktor, dessen Ausgabe vom aktuellen \textbf{Zustand} $z_n$ abhängig ist.
\end{definition}