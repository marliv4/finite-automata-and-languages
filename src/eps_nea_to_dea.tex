\subsubsection{$\epsilon$-NEA $\to$ DEA}
Es sei folgendes Zustandsdiagramm eines $\epsilon$-NEAs gegeben:\\
\begin{figure}[H]
    \centering
    \begin{transitiongraph}[fa]
        \state[s]{q0}{0}{0}
        \state[f]{q2}{60}{0}
        \state{q4}{120}{-60}
        \state{q1}{0}{-60}
        \state{q3}{120}{0}
        \transition[line=left]{q0}{q1}{A}
        \transition{q0}{q2}{}
        \transition{q2}{q3}{;B}
        \transition[line=left]{q2}{q4}{A}
        \transition[line=left]{q4}{q2}{B}
        \transition[line=left]{q1}{q0}{A}
        \transition{q3}{q4}{A}
    \end{transitiongraph}
    % \caption{epsNEA}
    \label{graph:epsNEA}
\end{figure}
Die Umwandlung in ein DEA geschieht wie üblich mit der Potenzmengenkonstruktion:
\begin{table}[H]
\centering
\begin{tabular}{|l|l|l|}
\hline
Zustand         & A           & B               \\ \hline
$\to q_0$       & $\{q_1,q_4\}$ & $\{q_3\}$           \\ \hline
$\{q_1,q_4\}$     & $\{q_0\}$     & $\{q_2\mbox{*}\}$ \\ \hline
$\{q_3\}$         & $\{q_4\}$     & $\emptyset$     \\ \hline
$\{q_2\mbox{*}\}$ & $\{q_4\}$     & $\{q_3\}$         \\ \hline
$\{q_4\}$         & $\emptyset$ & $\{q_2\}$           \\ \hline
$\emptyset$     & $\emptyset$ & $\emptyset$     \\ \hline
\end{tabular}
\end{table}
Anschlißend wird das neue Zustandsdiagramm des DEAs gezeichnet. \textit{qE} repräsentiert dabei die leere Menge\\
\begin{figure}[H]
    \centering
    \begin{transitiongraph}[fa]
        \state[s]{q0}{0}{0}
        \state{q1q4}{0}{-40}
        \state{q4}{80}{-80}
        \state[f]{q2}{40}{-40}
        \state{qE}{120}{0}
        \state{q3}{80}{0}
        \transition[line=left]{q0}{q1q4}{A}
        \transition{q0}{q3}{B}
        \transition[line=left]{q1q4}{q0}{A}
        \transition{q1q4}{q2}{B}
        \transition[line=left]{q4}{q2}{B}
        \transition{q4}{qE}{A}
        \transition[line=left]{q2}{q4}{A}
        \transition{q2}{q3}{B}
        \transition{qE}{qE}{A;B}
        \transition{q3}{q4}{A}
        \transition{q3}{qE}{B}
    \end{transitiongraph}
    \caption{Umwandlung von $\epsilon$-NEA zu DEA. Dieser ist jedoch nicht zwangsläufig optimal bzw. minimal.}
\end{figure}